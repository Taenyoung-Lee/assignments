\documentclass[12pt]{article}
\usepackage{amsmath, amssymb, amsthm, bm}
\usepackage{geometry}
\geometry{a4paper, margin=1in}
\usepackage{kotex}
\usepackage{hyperref}
\usepackage{mathtools}
\usepackage{enumitem}
\usepackage{booktabs}
\usepackage{graphicx}
\usepackage{float}
\usepackage{listings}

\lstset{
  basicstyle=\ttfamily\small,
  breaklines=true,
  columns=fullflexible,
  keepspaces=true,
  frame=single,
  language=R
}

\title{생존분석 시험대비 문제집 (문제 + 해설 포함)}
\author{Taenyoung 대비용 종합 세트}
\date{\today}

% -------- 문제/해설 환경 --------
\newtheoremstyle{myplain}%
  {6pt}{6pt}{\itshape}{}{\bfseries}{.}{0.5em}{}
\theoremstyle{myplain}
\newtheorem{problem}{문제}

\newenvironment{solution}{%
  \par\noindent\textbf{해설/정답.}\ }{\hfill$\square$\par}

% 약어
\newcommand{\Surv}{\mathrm{Surv}}
\newcommand{\Var}{\mathrm{Var}}
\newcommand{\E}{\mathrm{E}}

\begin{document}
\maketitle


\hrule
\vspace{0.6em}

\section{기초 및 연계식}

\begin{problem}[생존함수--위험함수 연계식 증명]
연속형 생존시간 $T$에 대해 $S(t)=P(T>t)$, $h(t)=\lim_{\Delta\to0} \frac{P(t<T\le t+\Delta\,|\,T>t)}{\Delta}$라 하자. 다음을 증명하라:
\[
S(t) = \exp\!\left(-\int_0^t h(u)\,du\right).
\]
\end{problem}

\begin{solution}
$f(t)=-S'(t)$이고 $h(t)=\frac{f(t)}{S(t)}=-\frac{S'(t)}{S(t)}=-(\ln S(t))'$. 적분하면
$\ln S(t)-\ln S(0)=-\int_0^t h(u)\,du$. $S(0)=1$이므로 $\ln S(0)=0$, 따라서
$S(t)=\exp\{-\int_0^t h(u)\,du\}$.
\end{solution}

\begin{problem}[Cox HR의 시간불변성]
Cox 모형 $h(t|X)=h_0(t)\exp(\beta^\top X)$에서 두 개인 $X,X^\ast$의 위험비(HR)가 시간에 의존하지 않음을 보여라.
\end{problem}

\begin{solution}
\[
\frac{h(t|X^\ast)}{h(t|X)}=\frac{h_0(t)\exp(\beta^\top X^\ast)}{h_0(t)\exp(\beta^\top X)}
=\exp\{\beta^\top(X^\ast-X)\},
\]
$h_0(t)$가 소거되어 시간 $t$에 무관.
\end{solution}

\section{KM, Greenwood, 로그순위}

\begin{problem}[KM 추정량과 Greenwood 분산(계산)]
사건시점 $t_{(1)}<\cdots<t_{(J)}$에서 위험집합 크기 $n_j$, 사건수 $d_j$가 주어질 때
\[
\hat S(t)=\prod_{t_{(j)}\le t}\left(1-\frac{d_j}{n_j}\right)
\]
의 표준오차 $\mathrm{SE}\{\hat S(t)\}$를 \textbf{Greenwood}로 구하는 공식을 쓰고, 아래 표에 대해 $\hat S(t_{(3)})$와 SE, 95\% CI(로그--로그 변환)를 계산하라.

\medskip
\begin{center}
\begin{tabular}{cccc}
\toprule
$j$ & $t_{(j)}$ & $n_j$ & $d_j$ \\
\midrule
1 & 2 & 100 & 3 \\
2 & 5 & 95 & 4 \\
3 & 8 & 90 & 6 \\
\bottomrule
\end{tabular}
\end{center}
\end{problem}

\begin{solution}
Greenwood:
\[
\widehat{\Var}\{\hat S(t)\}=\hat S(t)^2\sum_{t_{(j)}\le t}\frac{d_j}{n_j(n_j-d_j)},\quad
\mathrm{SE}=\sqrt{\widehat{\Var}}.
\]
수치:
\[
\hat S(t_{(3)})=\Big(1-\frac{3}{100}\Big)
\Big(1-\frac{4}{95}\Big)
\Big(1-\frac{6}{90}\Big)
=0.97\times 0.9579\times 0.9333\approx 0.868.
\]
분산합:
$\frac{3}{100\cdot 97}+\frac{4}{95\cdot 91}+\frac{6}{90\cdot 84}\approx 0.000309+0.000462+0.000794\approx 0.001565$.
따라서 $\widehat{\Var}\approx 0.868^2\times 0.001565\approx 0.00118$, SE $\approx 0.0343$.

로그--로그 변환 CI: $\eta=\log[-\log \hat S]$. $\hat S=0.868\Rightarrow -\log \hat S\approx 0.141$, $\eta\approx \log(0.141)\approx -1.958$.
표준오차는 델타법으로
$\mathrm{SE}_\eta\simeq \frac{1}{-\log \hat S}\cdot \frac{1}{\hat S}\cdot \mathrm{SE}_{\hat S}
\approx \frac{1}{0.141}\cdot \frac{1}{0.868}\cdot 0.0343\approx 0.283$.
95\% CI for $\eta$: $-1.958\pm 1.96\times 0.283\Rightarrow (-2.514,\,-1.402)$.
역변환: $S_L=\exp\{-\exp(\eta_U)\}$, $S_U=\exp\{-\exp(\eta_L)\}$.
$\exp(\eta_U)=\exp(-1.402)\approx 0.246\Rightarrow S_L\approx e^{-0.246}=0.782$.
$\exp(\eta_L)=\exp(-2.514)\approx 0.081\Rightarrow S_U\approx e^{-0.081}=0.922$.
따라서 95\% CI $\approx (0.782,\ 0.922)$.
\end{solution}

\begin{problem}[가중/층화 로그순위의 아이디어(서술)]
혼란변수 $Z$가 존재할 때 \textbf{층화 로그순위} 검정의 통계량 합성 원리를 간단히 서술하고,
Fleming--Harrington 계열과 같은 \textbf{가중 로그순위}의 직관을 2--3문장으로 답하라.
\end{problem}

\begin{solution}
층화 로그순위는 층 $s=1,\dots,S$에서 사건시점 $j$마다 관측--기대 차이 $O_{1,sj}-E_{1,sj}$를 구해
$U=\sum_s\sum_{j\in\mathcal{J}_s} w_{sj}(O_{1,sj}-E_{1,sj})$로 합산하고,
$\Var(U)=\sum_s\sum_{j\in\mathcal{J}_s} w_{sj}^2 V_{sj}$로 표준화한다.
가중 로그순위는 $w_{sj}$를 통해 \emph{초기/후기 사건}에 민감도를 조정한다.
예컨대 $w_j$가 생존함수의 함수(예: $S(t)^p(1-S(t))^q$)이면 $p$가 크면 \emph{초기}에, $q$가 크면 \emph{후기}에 더 민감해진다.
\end{solution}

\section{Cox: 부분가능도, ties, 추론}

\begin{problem}[부분가능도와 $h_0(t)$ 소거(서술)]
Cox 부분가능도의 로그형식
\[
\ell(\beta)=\sum_{j=1}^k\left\{\beta^\top x_{(j)}-\log\sum_{i\in R(t_{(j)})}\exp(\beta^\top x_i)\right\}
\]
이 \emph{조건부 확률의 곱}에서 나오며, $h_0(t)$가 소거되는 이유를 3--4문장으로 설명하라.
\end{problem}

\begin{solution}
$j$번째 사건시점에서 실패자의 \emph{위험비}가 위험집합 내에서 가장 먼저 실패할 조건부확률에 비례하며,
분자/분모 모두 $h_0(t_{(j)})$를 포함하나 동일 시점에서는 상쇄된다.
사건시점별 조건부확률을 곱하면 전체(부분)가능도가 되고, 로그를 취해 합으로 표현된다.
이 과정에서 $h_0(t)$ 모양은 필요치 않아 $\beta$ 추정이 가능하다.
\end{solution}

\begin{problem}[동일시각 사건 처리(ties) 개념 비교]
동일한 시점 $t$에 $d$건의 사건이 발생할 때 \textbf{Exact}, \textbf{Breslow}, \textbf{Efron} 방법의 차이를 2--3문장씩 요약하라.
\end{problem}

\begin{solution}
\emph{Exact}: $d$건의 모든 발생 순열을 고려한 정확 가능도. 계산량 큼, 소표본/동률 빈도 높을 때 정확.
\emph{Breslow}: $d$건을 동시에 발생한 것으로 간주, 분모를 $\big(\sum_{i\in R(t)} e^{\beta^\top x_i}\big)^d$로 근사. 단순하나 편향 가능.
\emph{Efron}: 위험집합에서 사건이 하나씩 제거되는 과정을 선형 보정해 분모를 조정. Exact에 더 근접, 실무 기본값으로 자주 사용.
\end{solution}

\section{PH 가정 점검: 로그--로그, 잔차, \texttt{cox.zph}}

\begin{problem}[로그--로그 평행성의 근거(요약)]
$S(t|X)=[S_0(t)]^{\exp(\beta^\top X)}$에서
\[
\log[-\log S(t|X)]=\beta^\top X + \log[-\log S_0(t)]
\]
임을 보이고, \emph{두 집단}의 로그--로그 곡선이 시간에 대해 \textbf{평행}해야 함을 설명하라.
\end{problem}

\begin{solution}
양변 로그 후 재로그로 위 식이 도출된다. 두 집단 $X_1,X_2$의 차이는
$\big(\beta^\top X_1-\beta^\top X_2\big)$라는 \emph{상수}이고 $t$에 무관하므로 평행성 신호가 된다.
\end{solution}

\begin{problem}[Schoenfeld 잔차와 \texttt{cox.zph} 전역검정]
Schoenfeld 잔차 $r_k=x_{(k)}-\bar x(\hat\beta,t_{(k)})$의 정의와 ``PH가 성립하면 시간과 비상관''인 이유를 설명하고,
\texttt{cox.zph}에서 \texttt{transform="identity", "km", "rank"}의 목적을 1--2문장씩 쓰며, \textbf{전역(Global)} p값 해석을 한 줄로 요약하라.
\end{problem}

\begin{solution}
$r_k$는 $k$번째 사건에서 관측 공변량과 위험집합 가중평균의 차이다.
PH가 성립하면 계수효과가 시간에 일정하여 잔차가 시간에 체계적 패턴을 보이지 않으므로 \emph{비상관}이어야 한다.
\texttt{identity}: 원시(선형) 시간, \texttt{km}: Kaplan--Meier 기반 변환으로 말단에서 변동 안정화, \texttt{rank}: 사건순위로 비모수적 시간척도.
전역 p값은 \emph{모든 공변량}에 대해 PH 가정이 동시에 성립하는지의 귀무가설 검정 결과다.
\end{solution}

\section{조정 생존곡선(Observed vs Expected)}

\begin{problem}[Breslow $\hat H_0(t)$와 조정 생존곡선]
Breslow 누적기저위험
\[
\hat H_0(t)=\sum_{t_{(j)}\le t}\frac{d_j}{\sum_{i\in R(t_{(j)})}\exp(\hat\beta^\top x_i)},
\quad \hat S_0(t)=\exp\{-\hat H_0(t)\}
\]
을 쓰고, 임의 공변량 $x$에 대한 조정 생존함수 $\hat S(t|x)=\hat S_0(t)^{\exp(\hat\beta^\top x)}$를 도출하라.
또한 \emph{Observed(KM)}과 \emph{Expected(조정곡선)}을 비교해 PH 진단에 어떻게 쓰는지 2--3문장으로 설명하라.
\end{problem}

\begin{solution}
정의 그대로이며, Cox 모형에서 $S(t|x)=S_0(t)^{\exp(\beta^\top x)}$ 관계를 추정치로 대체한다.
PH가 성립하면 관찰된 KM과 모형기반 조정곡선이 체계적으로 벌어지지 않는다.
일정 구간에서 일관된 편차가 있다면 시간가변 효과 또는 모형 부적합 신호로 본다.
\end{solution}

\section{잔차/영향도 (Martingale/Deviance/dfbeta)}

\begin{problem}[잔차와 영향점 진단(핵심 요약)]
Martingale 잔차, Deviance 잔차, dfbeta(dfbetas)의 용도와 해석 포인트를 각 1--2문장으로 요약하라.
\end{problem}

\begin{solution}
Martingale: $M_i=\delta_i-\hat H_0(T_i)\exp(\hat\beta^\top x_i)$로 함수형/선형성 위반 탐지.
Deviance: $r_{Di}=\mathrm{sign}(M_i)\sqrt{-2(M_i+\delta_i\log(\delta_i-M_i))}$로 극단 관측치 탐지에 유용.
dfbeta(dfbetas): 개별 관측 제거 시 $\hat\beta$ 변화량(표준화)을 측정, 영향력이 큰 관측을 식별한다.
\end{solution}

\section{R 실습 종합(스캐폴드)}

\begin{problem}[실습: KM/로그순위/층화/가중, Cox/ties, \texttt{cox.zph}, 조정곡선]
다음 스캐폴드를 이용해 데이터프레임 \texttt{df}(\texttt{time}, \texttt{status}, \texttt{group}, \texttt{x1}, \texttt{x2}, \texttt{Z})에 대해
(i) KM+로그순위(표준/층화/가중), (ii) Cox(ties: efron/breslow/exact), (iii) \texttt{cox.zph}(변환 3종, 전역 p값), (iv) 조정 생존곡선을 수행하라.
핵심 수치결과(계수/HR/CI/logLik/p값)를 표로 요약하고, PH 진단 결과를 5--7줄로 해석하라.
\end{problem}

\begin{solution}


\textbf{보고/해석 템플릿 예시}:
\begin{itemize}[leftmargin=*]
\item 로그-순위(표준) p값 $\approx p_0$, 층화 후 $p_{\mathrm{strat}}$로 변화. (혼란 조정에 따라 결론 변화 여부 기술)
\item Cox(efron) 추정: HR(group B vs A)$=\exp(\hat\beta_{\text{group}})$, 95\% CI, Wald p값 보고.
\item ties 방법별 로그가능도/계수 비교: Efron과 Exact가 유사, Breslow는 (데이터 특성에 따라) 경향 차이.
\item \texttt{cox.zph}: 변수별 p값과 \emph{전역} p값. 변환 3종에서 결론 일관성 여부 확인.
\item 조정 생존곡선 vs KM: 체계적 벌어짐이 있으면 PH 위반/함수형 문제 가능성 코멘트.
\end{itemize}
\end{solution}

\section{확장 Cox: 시간의존 효과}

\begin{problem}[시간의존 상호작용 검정]
확장 Cox $h(t|X)=h_0(t)\exp\{\beta X+\delta X g(t)\}$에서 $H_0:\delta=0$의 의미를 쓰고,
$g(t)=\log t$일 때 \texttt{R}에서 \texttt{tt=}을 이용해 구현하는 스니펫을 제시하라.
\end{problem}

\begin{solution}
$H_0:\delta=0$은 $X$의 효과가 \emph{시간불변}임(=PH 성립)을 뜻한다.
유의한 $\hat\delta\neq 0$이면 시간가변 HR을 시사.
\begin{lstlisting}
coxph(Surv(time, status) ~ X + tt(X), data=df,
      tt = function(t, x, ...) { x * log(pmax(t, 1) ) })
\end{lstlisting}
\end{solution}

\bigskip
\hrule
\bigskip

\section*{추가: 점수 올리는 한 줄 메모}
\begin{itemize}[leftmargin=*]
\item \textbf{KM 신뢰구간}: 로그--로그 변환을 권장(대칭성 개선).
\item \textbf{가중 로그-순위}: 초기/후기 민감도는 가중식으로 제어.
\item \textbf{ties}: 실무 기본은 \texttt{efron}; 동률 다발/소표본이면 \texttt{exact} 검토.
\item \textbf{PH 진단}: 변수별 + 전역(Global) p값 \emph{둘 다} 보고.
\item \textbf{조정곡선}: Observed(KM) vs Expected(조정) 체계적 벌어짐 여부 확인.
\item \textbf{잔차/영향}: Martingale(함수형), Deviance(극단치), dfbetas(영향점) 역할 구분.
\end{itemize}

\end{document}
