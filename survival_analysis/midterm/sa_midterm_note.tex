\documentclass{article}
\usepackage{amsmath}
\usepackage{amssymb}
\usepackage{graphicx}
\usepackage{geometry}
\geometry{a4paper, margin=1in}
\usepackage{hyperref}
\usepackage{kotex}

\title{시험 대비 생존 분석 심화 개념 정리 (증명 포함)}
\author{}
\date{}

\begin{document}

\maketitle

\section{생존 분석의 기초}

생존 분석은 어떤 사건(event)이 발생하기까지 걸리는 시간 변수와 이에 영향을 미치는 요인들을 통계적으로 분석하는 방법론입니다.

\subsection{주요 함수와 그 관계}

\begin{itemize}
    \item \textbf{생존 함수 (Survivor Function, S(t)):} 특정 시점 $t$를 지나 생존할 확률입니다.
    $$S(t) = P(T > t), \quad t > 0$$

    \item \textbf{위험 함수 (Hazard Function, h(t)):} 시점 $t$까지 생존했을 때, 바로 그 순간($t$와 $t+\Delta t$ 사이)에 사건이 발생할 조건부 확률을 나타내는 순간 위험률입니다.
    $$h(t) = \lim_{\Delta t \to 0} \frac{P(t < T \le t + \Delta t | T > t)}{\Delta t}$$

    \item \textbf{S(t)와 h(t)의 관계 증명:}
    위험 함수의 정의에 따라, $h(t)$는 다음과 같이 표현될 수 있습니다.
    $$h(t) = \frac{f(t)}{S(t)}$$
    여기서 $f(t)$는 시점 $t$에서의 사건 발생 확률 밀도 함수(PDF)이며, $f(t) = -S'(t) = -\frac{dS(t)}{dt}$ 입니다. 따라서,
    $$h(t) = -\frac{S'(t)}{S(t)} = -\frac{d}{dt} \ln S(t)$$
    이 식을 0부터 $t$까지 적분하면,
    $$\int_0^t h(u)du = -\int_0^t \frac{d}{du} \ln S(u)du = -[\ln S(t) - \ln S(0)]$$
    $S(0)=P(T>0)=1$이므로 $\ln S(0)=0$ 입니다. 따라서,
    $$\int_0^t h(u)du = -\ln S(t)$$
    이를 $S(t)$에 대해 정리하면 다음과 같은 핵심 관계식이 유도됩니다.
    $$S(t) = \exp\left[-\int_0^t h(u)du\right]$$
\end{itemize}

\section{카플란-마이어(KM) 곡선과 로그-순위 검정}

\subsection{카플란-마이어 (Kaplan-Meier) 추정량}
KM 추정량은 중도절단이 있는 데이터에서 생존 함수를 추정하는 비모수적 방법입니다.

\begin{itemize}
    \item \textbf{KM 공식 (Product-Limit Formula):}
    시점 $t_{(j)}$에서의 생존율은 그 이전 시점까지의 생존율에 해당 시점에서 생존할 조건부 확률을 곱하여 계산됩니다.
    $$\hat{S}(t) = \prod_{t_{(i)} \le t} \left( \frac{n_i - m_i}{n_i} \right)$$
    여기서 $t_{(i)}$는 $i$번째 사건 발생 시점, $n_i$는 해당 시점의 위험 집단(risk set) 크기, $m_i$는 해당 시점의 사건 수입니다.

    \item \textbf{KM 공식의 유도:}
    생존 함수의 정의에 따라, 시점 $t_{(j)}$를 지나 생존하는 것은 시점 $t_{(1)}, t_{(2)}, \dots, t_{(j)}$에서 연속적으로 사건을 겪지 않는 것과 같습니다. 이는 조건부 확률의 곱으로 표현할 수 있습니다.
    $$\hat{S}(t_{(j)}) = \prod_{i=1}^{j} P(T > t_{(i)} | T \ge t_{(i)})$$
    $i$번째 사건 시점에서 생존할 조건부 확률은 $(n_i - m_i)/n_i$로 추정되므로, 위 공식이 유도됩니다. 또한, 이는 재귀적으로도 표현 가능합니다:
    $$\hat{S}(t_{(j)}) = \hat{S}(t_{(j-1)}) \times P(T > t_{(j)} | T \ge t_{(j)})$$
\end{itemize}

\subsection{로그-순위 검정 (Log-Rank Test)}
로그-순위 검정은 여러 그룹의 생존 곡선이 통계적으로 동일한지를 검정하는 방법입니다.

\begin{itemize}
    \item \textbf{귀무가설 ($H_0$):} 모든 그룹의 생존 곡선은 같다 ($S_1(t) = S_2(t) = \dots = S_G(t)$).
    \item \textbf{원리:} 각 사건 발생 시점 $j$마다 그룹 1의 \textbf{기대 사건 수($e_{1j}$)}를 계산합니다. 이는 해당 시점의 전체 사건 수($m_j = m_{1j} + m_{2j}$)를 위험 집단의 비율에 따라 배분한 값입니다.
    $$e_{1j} = (m_{1j} + m_{2j}) \times \left( \frac{n_{1j}}{n_{1j} + n_{2j}} \right)$$
    \item \textbf{검정 통계량:} 관측된 총 사건 수($O_i$)와 기대된 총 사건 수($E_i$)의 차이를 이용합니다. 두 그룹 비교 시, 검정 통계량은 다음과 같으며 자유도가 1인 카이제곱 분포를 따릅니다.
    $$\text{Log-Rank Statistic} = \frac{(O_2 - E_2)^2}{\text{Var}(O_2 - E_2)} \sim \chi^2(1)$$
    $G$개 그룹 비교 시에는 자유도가 $G-1$인 카이제곱 분포를 따릅니다.
\end{itemize}

% --------- 보강 ①: 층화/가중 로그-순위 검정 ----------
\paragraph{(보강) 층화/가중 로그-순위 검정}
공변량(예: 검사치 범주)을 층화변수 $s=1,\dots,S$로 나누고, 각 사건시점의 관측/기대 사건수 차이를 층별로 계산한 후 합산한다.
\[
U = \sum_{s=1}^S \sum_{j \in \mathcal{J}_s} w_{sj}\,[O_{1,sj}-E_{1,sj}], 
\quad
Var(U)=\sum_{s=1}^S \sum_{j \in \mathcal{J}_s} w_{sj}^2\,V_{sj},
\]
여기서 $w_{sj}$는 가중치(기본 로그-순위는 $w_{sj}=1$), $\mathcal{J}_s$는 층 $s$의 사건시점 집합이다. 검정통계량 $Z=U/\sqrt{Var(U)}$ 또는 $\chi^2=Z^2$로 유의성을 판단한다. 
실무에서는 \texttt{survdiff}의 \texttt{rho}로 $w_{sj}$를 달리할 수 있다(예: Fleming--Harrington 계열; $\rho=0$가 표준 로그-순위).
또한 공변량에 따라 층화하려면 \texttt{survdiff(time \textasciitilde{} group + strata(z))}처럼 \texttt{strata()}를 사용한다.
% --------------------------------------------------------

\section{콕스 비례위험(Cox Proportional Hazards) 모델}

\subsection{콕스 PH 모델의 공식과 가정}
콕스 PH 모델은 공변량이 생존 시간에 미치는 영향을 분석하는 준모수적 회귀 모델입니다.

\begin{itemize}
    \item \textbf{모델 공식:}
    $$h(t, X) = h_0(t) \cdot \exp\left(\sum_{i=1}^{p} \beta_i X_i\right)$$
    여기서 $h_0(t)$는 \textbf{기저 위험 함수(baseline hazard function)}로, 형태를 특정하지 않아 비모수적(non-parametric) 부분이며, $\exp(\dots)$ 부분은 모수적(parametric) 부분입니다.

    \item \textbf{핵심 가정 (비례 위험 가정):} 두 공변량 벡터 $X^*$와 $X$를 가진 개인들의 \textbf{위험비(Hazard Ratio)}는 시간에 따라 변하지 않고 일정하다는 것입니다.
    $$\frac{h(t, X^*)}{h(t, X)} = \text{constant for all } t$$
\end{itemize}

\subsection{위험비 (Hazard Ratio, HR)의 유도}
위험비는 두 그룹의 위험 함수 간의 비율로, 콕스 모델에서 다음과 같이 유도됩니다.
$$\text{HR} = \frac{h(t, X^*)}{h(t, X)} = \frac{h_0(t) \exp(\sum \beta_i X_i^*)}{h_0(t) \exp(\sum \beta_i X_i)} = \exp\left[\sum \beta_i (X_i^* - X_i)\right]$$
여기서 기저 위험 함수 $h_0(t)$가 소거되므로, 위험비는 시간에 의존하지 않는 상수가 됩니다.

\subsection{부분 가능도 (Partial Likelihood)}
콕스 모델의 회귀 계수 $\beta$는 \textbf{부분 가능도(Partial Likelihood)}를 최대화하여 추정됩니다. 이는 각 사건 발생 시점에서, 실제로 사건을 겪은 개인이 위험 집단 내 다른 이들보다 먼저 사건을 겪을 조건부 확률들의 곱으로 구성됩니다.
$j$번째 사건이 발생했을 때의 가능도 $L_j$는 다음과 같습니다.
$$L_j(\beta) = \frac{\text{Hazard for the individual who fails at } t_{(j)}}{\sum_{k \in R(t_{(j)})} \text{Hazards for all individuals in risk set } R(t_{(j)})}$$
$$L_j(\beta) = \frac{h_0(t_{(j)}) \exp(\sum \beta_i X_{ij})}{ \sum_{k \in R(t_{(j)})} h_0(t_{(j)}) \exp(\sum \beta_i X_{ik})} = \frac{\exp(\sum \beta_i X_{ij})}{\sum_{k \in R(t_{(j)})} \exp(\sum \beta_i X_{ik})}$$
전체 부분 가능도는 모든 사건에 대한 $L_j$의 곱입니다: $L(\beta) = \prod_{j=1}^k L_j(\beta)$. 여기서도 $h_0(t)$가 소거되므로, 기저 위험 함수를 몰라도 $\beta$를 추정할 수 있습니다.

% --------- 보강 ②: ties 처리 ----------
\paragraph{(보강) 동일시각 사건(ties)의 처리}
동일한 시점 $t$에 사건이 $d$건 발생하면 부분가능도의 분모/분자 정의가 달라진다.
대표적 방법은 다음과 같다.
\begin{itemize}
  \item \textbf{정확법(Exact/Marginal)}: $d$건의 사건 발생 순열을 모두 고려한 정확 가능도를 사용. 계산량이 크나 가장 정확.
  \item \textbf{Breslow 근사}: 사건 $d$건을 한꺼번에 발생한 것으로 보고 분모를 $\big(\sum_{i\in R(t)} e^{\beta^\top x_i}\big)^d$로 근사.
  \item \textbf{Efron 근사}: $d$건이 위험집합에서 점진적으로 제거된다고 보고 분모에 보정항을 도입(정확법에 더 근접).
\end{itemize}
실무에선 표본크기/동률 빈도에 따라 \texttt{coxph(..., ties="efron")} (기본), \texttt{"breslow"}, \texttt{"exact"}를 선택한다.
% ---------------------------------------

\section{비례 위험(PH) 가정 평가}

\subsection{로그-로그 플롯(Log-log Plots)의 원리 증명}
로그-로그 플롯이 평행해야 하는 이유는 콕스 모델의 생존 함수로부터 수학적으로 유도됩니다.

\begin{enumerate}
    \item 콕스 모델의 생존 함수는 다음과 같습니다:
    $$S(t, X) = [S_0(t)]^{\exp(\sum \beta_i X_i)}$$
    여기서 $S_0(t)$는 기저 생존 함수입니다.

    \item 양변에 자연로그를 취합니다 (Log \#1):
    $$\ln S(t, X) = \exp\left(\sum \beta_i X_i\right) \cdot \ln S_0(t)$$

    \item 양변에 음수를 곱하고 다시 자연로그를 취합니다 (Log \#2):
    $$\ln[-\ln S(t, X)] = \ln\left[-\exp\left(\sum \beta_i X_i\right) \cdot \ln S_0(t)\right]$$
    $$= \ln\left(\exp\left(\sum \beta_i X_i\right)\right) + \ln(-\ln S_0(t))$$
    $$= \sum \beta_i X_i + \ln(-\ln S_0(t))$$

    \item 두 개인(공변량 $X_1$, $X_2$)에 대한 로그-로그 생존 함수의 차이를 계산하면 다음과 같습니다:
    $$\ln[-\ln S(t, X_1)] - \ln[-\ln S(t, X_2)] = \left(\sum \beta_i X_{1i} + \ln(-\ln S_0(t))\right) - \left(\sum \beta_i X_{2i} + \ln(-\ln S_0(t))\right)$$
    $$= \sum \beta_i (X_{1i} - X_{2i})$$
    결과적으로, 두 로그-로그 플롯 간의 수직 거리는 시간에 의존하지 않는 상수($\sum \beta_i (X_{1i} - X_{2i})$)가 됩니다. 따라서 두 그래프는 평행해야 합니다.
\end{enumerate}

\subsection{시간 의존 변수를 이용한 검정}
PH 가정을 통계적으로 검정하는 가장 엄격한 방법은 시간 의존 변수를 포함하는 \textbf{확장된 콕스 모델(extended Cox model)}을 사용하는 것입니다.
$$h(t, X) = h_0(t) \exp[\beta X + \delta (X \times g(t))]$$
여기서 $g(t)$는 시간의 함수(예: $t$ 또는 $\ln(t)$)입니다.
\begin{itemize}
    \item \textbf{귀무가설 ($H_0$): $\delta = 0$}.
    \item 만약 귀무가설이 기각되면($\delta$가 0과 유의하게 다르면), $X$의 효과가 시간에 따라 변한다는 의미이므로, $X$는 PH 가정을 위배한 것입니다.
\end{itemize}

% ------------------ 2.4 보강 블록 ------------------
\subsection{PH 가정의 그래프적/잔차 기반 점검과 실무적 대처}

\paragraph{(1) Observed vs Expected 비교(조정 생존곡선)}
콕스모형 적합 후 각 사건시점 $t_{(j)}$에서 집단 $g$의 \emph{기대 사건수}는
\[
e_{gj} \;=\; d_j\,
\frac{\sum_{i\in R(t_{(j)})\cap g} \exp(\hat\beta^\top x_i)}{\sum_{i\in R(t_{(j)})} \exp(\hat\beta^\top x_i)},
\]
여기서 $d_j$는 $t_{(j)}$에서의 총 사건 수, $R(t_{(j)})$는 위험집합입니다.
누적 관측--기대 차이
\[
C_g(t) \;=\; \sum_{t_{(j)}\le t}\{O_{gj}-e_{gj}\}
\]
를 시점에 따라 그리거나, \emph{조정 생존곡선}
\[
\hat H_0(t)=\sum_{t_{(j)}\le t}\frac{d_j}{\sum_{i\in R(t_{(j)})}\exp(\hat\beta^\top x_i)},\quad
\hat S_0(t)=\exp\{-\hat H_0(t)\},\quad
\hat S(t\,|\,x)=\hat S_0(t)^{\exp(\hat\beta^\top x)}
\]
을 생성하여 집단별 대표 공변량(또는 평균/중앙 공변량)으로 비교합니다.
PH가 성립하면 관찰(KM)과 모형기반(조정) 곡선이 체계적으로 벌어지지 않습니다.

\paragraph{(2) Schoenfeld 잔차 기반 PH 가정 검정(Grambsch--Therneau)}
$k$번째 사건시점에서 실패한 개인의 공변량을 $x_{(k)}$라 하고, 위험집합 가중 평균을
\[
\bar x(\beta,t_{(k)}) \;=\; 
\frac{\sum_{i\in R(t_{(k)})} x_i \exp(\beta^\top x_i)}
{\sum_{i\in R(t_{(k)})} \exp(\beta^\top x_i)}
\]
로 두면, \emph{Schoenfeld 잔차}는
\[
r_k \;=\; x_{(k)} - \bar x(\hat\beta,t_{(k)}).
\]
PH가 성립하면 $r_k$는 시간에 의존적 패턴을 갖지 않습니다.
실무에서는 \emph{scaled} 잔차 $\tilde r_k$(정보행렬로 정규화)를 구성하여, 각 공변량별로 $\tilde r_k$를 사건시간의 함수(예: $\log t$ 또는 순위)에 회귀시켜 \emph{기울기=0}을 검정합니다.
변수별 p값과 전역(global) p값을 함께 보고, 유의하면 해당 공변량의 PH 위반을 시사합니다.

% --------- 보강 ③: cox.zph 구현 디테일 ----------
\paragraph{(보강) \texttt{cox.zph}의 변환 선택과 전역(Global) 검정}
Schoenfeld(및 scaled) 잔차에 대해 \texttt{cox.zph}는 시간축 변환으로 \texttt{"identity"}, \texttt{"km"}, \texttt{"rank"} 등을 제공한다.
변환은 잔차--시간 관계의 선형화를 돕는 목적이며, 변수별 p값과 함께 \textbf{전역(Global) p값}으로 모형 전반의 PH 가정을 평가한다.
여러 변환에서 유사한 결론이 나오는지 함께 확인하면 해석이 안정적이다.
% ---------------------------------------------------

\paragraph{(3) 시간의존 상호작용으로 엄밀 검정(확장 Cox)}
이미 기술한 확장 콕스
\[
h(t\,|\,X)=h_0(t)\exp\{\beta X+\delta\,X\,g(t)\}
\]
에서 $H_0:\delta=0$을 Wald/Score/LRT로 검정합니다.
$g(t)$는 $\log t$, $t$, 구간함수 등으로 두며, 유의한 $\delta$는 $X$ 효과가 시간에 따라 변함을 의미(=PH 위반)합니다.

\paragraph{(4) PH 위반 시 대처 전략}
\begin{itemize}
  \item \textbf{층화 Cox}: 문제가 되는 범주형 공변량으로 층화하여 $h_0(t)$를 층별로 분리(해당 변수의 HR은 추정하지 않음).
  \item \textbf{구간별 HR(piecewise)}: $t$를 구간으로 나눠 $X\times I(t\in\text{구간})$로 효과를 시점별로 다르게 허용.
  \item \textbf{시간의존 효과 모델링}: $X\times g(t)$의 함수형을 모형 안에 명시(유도한 $\delta$ 포함).
  \item \textbf{모형 변경 고려}: PH 가정이 본질적으로 맞지 않는다면 AFT 등 대안 모형 검토.
\end{itemize}


% ------------------ 2.4 보강 블록 끝 ------------------

\end{document}
