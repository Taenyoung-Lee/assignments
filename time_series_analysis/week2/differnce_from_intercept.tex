\documentclass[12pt]{article}
\usepackage{bm}
\usepackage{amsmath, amssymb} % 수식
\usepackage{geometry}
\usepackage{kotex}

\geometry{margin=1in}

\title{지시변수와 연속형 변수가 있을 때 \texttt{NOINT}의 효과}
\author{}
\date{}

\begin{document}
\maketitle

\section*{1. 표기와 가정}

관측이 $t=1,\dots,n$개 있다고 하자. 상태(범주)가 $k$개이며, 각 상태에 대한 지시변수를
\[
I_j(t) = \mathbf{1}\{\text{state}(t)=j\}, \qquad j=1,\dots,k
\]
로 두고, 지시행렬을 $D = [\bm d_1, \bm d_2, \dots, \bm d_k]$라 한다.
모든 관측은 정확히 한 상태에 속한다고 가정한다(one-hot).

연속형 설명변수 행렬은 $Z \in \mathbb{R}^{n\times q}$, 상수항은 $\bm 1_n$으로 둔다.

\section*{2. 지시변수만 있을 때}

각 행은 한 열만 $1$이고 나머지는 $0$이므로
\[
\bm d_1 + \cdots + \bm d_k = \bm 1_n.
\]
따라서 상수항은 이미 $D$의 열공간에 포함된다.
\[
\mathcal C([\,\bm 1_n, D_{-k}]) = \mathcal C(D).
\]
따라서 절편을 포함하거나 \texttt{NOINT}를 쓰거나 적합값과 자유도는 동일하다.
차이는 계수 해석뿐이다:
\[
\beta_j^{(\text{NOINT})} = \text{셀 평균}, \qquad
\alpha, \gamma_j = \text{기준 대비 효과}.
\]

\section*{3. 지시변수 + 연속형 변수가 있는 경우}

설계행렬은
\[
X_{\mathrm{int}} = [\,\bm 1_n \;|\; D_{-k}\;|\; Z\,], \qquad
X_{\mathrm{no}} = [\,D\;|\;Z\,].
\]

상수항 $\bm 1_n$은 $D$의 열공간에 속하므로
\[
\mathcal C(X_{\mathrm{int}}) = \mathcal C([D, Z]) = \mathcal C(X_{\mathrm{no}}).
\]
즉, 적합값과 SSE, 자유도는 동일하다.
\texttt{NOINT} 모형은 상태별 평균을 직접 계수로 두고,
절편 포함 모형은 기준 대비 효과와 절편으로 표현한다.

계수 사이의 대응은
\[
\beta_k = \alpha, \qquad
\beta_j = \alpha + \gamma_j \quad (j=1,\dots,k-1),
\]
그리고 $Z$의 계수는 두 모형에서 동일하다.

\section*{4. 자유도와 랭크}

잔차 자유도는
\[
\mathrm{df} = n - \operatorname{rank}(X).
\]
$D$가 완전(one-hot)이라면 $\operatorname{rank}(D)=k$이고,
두 모형 모두 랭크가
\[
k + \operatorname{rank}\big((I - P_D)Z\big)
\]
이므로 df도 같다.

\section*{5. 시계열 맥락에서의 해석}

\begin{itemize}
\item \textbf{지시변수만 있는 경우:} 계절 더미나 구간 효과 모형에서 \texttt{NOINT}는 각 구간 평균을 직접 추정한다. 적합값, SSE, df가 모두 동일.
\item \textbf{지시 + 연속형 변수:} 지시행렬이 완전하면, 연속형 변수가 있더라도 설계공간은 같고 결과는 동일하다. 다만, 계수 해석은 ``상태별 절편'' vs ``기준대비 효과''로 달라진다.
\end{itemize}

\section*{6. 주의할 점}

적합값이 동일하다는 결론은 다음이 만족될 때만 성립한다.
\begin{itemize}
\item 지시행렬이 완전(one-hot)이어야 함.
\item \texttt{NOINT} 모형에서 지시열을 빠뜨리지 않아야 함.
\item 오차 구조가 동일해야 함(예: OLS).
\end{itemize}
이 중 하나라도 깨지면, 두 모형의 설계공간이 달라지고
적합값, SSE, 자유도가 달라질 수 있다.

\section*{7. 요약}

\begin{itemize}
\item 지시변수만 있을 때: \texttt{NOINT} 여부는 적합값에 영향을 주지 않는다.
\item 지시 + 연속형 변수가 함께 있어도, 지시행렬이 완전하면 결과는 동일하며,
      해석(계수 표현)만 다르다.
\item 지시행렬이 완전하지 않거나 일부 열을 제거하면 결과가 달라질 수 있다.
\end{itemize}

\end{document}
