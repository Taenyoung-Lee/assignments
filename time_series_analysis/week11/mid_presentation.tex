\documentclass{beamer}

% ===== 기본 설정 =====
\usetheme{Madrid}
\usecolortheme{default}

% [demo] 옵션은 이미지가 없을 때 검은 상자로 대체하여 에러를 방지합니다.
% 실제 이미지 파일(tsplot.pdf 등)이 준비되어 있다면 [demo]를 지우세요.
\usepackage{graphicx} 

\usepackage{kotex}          % 한글
\usepackage{amsmath,amssymb,bm}
\usepackage{booktabs}
\usepackage{mathtools}

\title[강릉 강수량 시계열]{강릉 월별 강수량 시계열 모형과\\조건부 저강수 월 탐지}
\author[Taenyoung Lee]{Taenyoung Lee}
\date[2025-11-24]{2025년 11월 24일}

%===========================================================
\begin{document}
%===========================================================

\begin{frame}
  \titlepage
\end{frame}

\begin{frame}{발표 개요}
  \tableofcontents
\end{frame}

%===========================================================
\section{문제 설정과 자료}
%===========================================================

\begin{frame}{연구 질문과 아이디어}
  \begin{itemize}
    \item 자료: 강릉 월별 강수량 (1998-01 $\sim$ 2025-09, monthly)
    \item 목표
      \begin{itemize}
        \item 계절성을 반영한 시계열 모형(SARIMA, ARMA + 계절더미) 적합
        \item 시계열적인 \alert{이상치(outlier)} 자동 탐지
        \item 모형이 기대하는 값보다 \alert{유의하게 적게 내린 달}을
              \emph{조건부 저강수 월}로 정의
      \end{itemize}
    \item 오늘 발표에서 할 것
      \begin{enumerate}
        \item 기본 SARIMA 모형과 잔차 진단
        \item ARMA + 계절더미 모형과 잔차 진단
        \item tsoutliers를 이용한 이상치 탐지 결과
        \item SARIMA/ARMA+계절더미 기반 조건부 저강수 월 정의
        \item 현재까지의 한계와 앞으로의 방향
      \end{enumerate}
  \end{itemize}
\end{frame}

\begin{frame}{자료 요약 및 원시계열}
  \begin{itemize}
    \item 시작: 1998-01, 종료: 2025-09, 주기: 12 (월별 시계열)
  \end{itemize}
  \vfill
  \begin{center}
    \includegraphics[width=0.9\textwidth]{tsplot.png}
  \end{center}
  \vspace{0.3em}
  \small
  \centering{``강릉 월별 강수량(원자료)'' 선그래프}
\end{frame}

%===========================================================
\section{기본 SARIMA 모형}
%===========================================================

\begin{frame}{정상성 점검과 모형 후보 탐색}
  \begin{itemize}
    \item 원자료 ACF/PACF에서 \alert{강한 12개월 주기성} 확인
    \item 12-계절차분 $y_t^\ast = (1 - B^{12})y_t$ 적용 후 ACF/PACF로
          계절 MA(1) 구조가 적당해 보임
  \end{itemize}
  \vfill
  \begin{columns}[T]
    \column{0.5\textwidth}
      \begin{center}
        \includegraphics[width=\textwidth]{plot_origin_acf.png}
        \small ACF / PACF (원자료)
      \end{center}
    \column{0.5\textwidth}
      \begin{center}
        \includegraphics[width=\textwidth]{plot_acf.png}
        \small ACF / PACF (12-계절차분 후)
      \end{center}
  \end{columns}
  \vfill
  \begin{itemize}
    \item AIC 그리드 탐색: $p=0\ldots3,\ q=0\ldots3,\ P=0\ldots2,\ Q=0\ldots2,\ D=1$에서
          \[
            \text{ARIMA}(2,0,2)(0,1,1)_{12}
          \]
          모형이 가장 작은 AIC를 가짐 
  \end{itemize} % <<< 여기가 누락되었었습니다.
\end{frame}   % <<< 여기가 누락되었었습니다.

\begin{frame}{최종 SARIMA 모형과 식}
  \begin{block}{선택된 기본 모형}
    \[
      \text{ARIMA}(2,0,2)(0,1,1)_{12},\quad \text{(절편 없이)} 
    \]
  \end{block}
\end{frame}

%===========================================================
\section{시계열 이상치 탐지}
%===========================================================

\begin{frame}{tsoutliers를 이용한 이상치 탐지}
  \begin{itemize}
    \item \texttt{forecast::tsoutliers(y)}로 자동 이상치 탐지
    \item 결과: 시점 index
      \[
        \{56,\ 93,\ 103,\ 262,\ 273\}
      \]
      등 14개가 이상치로 탐지됨
    \item 월 단위로 환산하면
      \begin{itemize}
        \item 56: 2002-08
        \item 93: 2005-09
        \item 103: 2006-07
        \item 262: 2019-10
        \item 273: 2020-09
      \end{itemize}
    \item 모두 \alert{강수량이 비정상적으로 큰 이벤트}로 해석 가능
  \end{itemize}
\end{frame}
\begin{frame}
    
  \begin{center}
    \includegraphics[width=0.8\textwidth]{plot_outlier.png}
    \small 빨간 점: tsoutliers가 탐지한 이상치 월
  \end{center}
\end{frame}

%===========================================================
\section{조건부 저강수 월 정의}
%===========================================================

\begin{frame}{아이디어: ``조건부 저강수 월''}
  \begin{block}{동기}
    \begin{itemize}
      \item 단순히 ``강수량이 적은 달''이 아니라,
      \item \emph{모형이 기대한 값보다} 유의하게 적게 내린 달을 잡고 싶다.
      \item 계절성 + 자기상관 구조를 제거한 후 남는 ``이상하게 적은 부분''에 관심.
    \end{itemize}
  \end{block}
  \vfill
  \begin{block}{제안}
    \begin{itemize}
      \item 모형으로 얻은 조건부 기대값 $\hat{y}_t = \mathbb{E}(y_t \mid \text{과거})$
      \item 예측오차의 표준편차 $\hat{\sigma}$를 이용하여
        \[
          z_t = \frac{y_t - \hat{y}_t}{\hat{\sigma}}
        \]
      \item 임계값 $k$에 대해
        \[
          z_t < -k \quad\Rightarrow\quad
          \text{``조건부 저강수 월''로 판단}
        \]
      \item 실제 구현에서는 $k = 1.5$를 사용 (조정 가능)
    \end{itemize}
  \end{block}
\end{frame}

\begin{frame}{모형 기반 조건부 저강수 월 예시}
  \begin{columns}[T]
    \column{0.48\textwidth}
      \begin{center}
        \includegraphics[width=\textwidth]{plot_arima_z.png}
        \small $z_t$ 시계열과 $-k$ 기준선
      \end{center}
    \column{0.48\textwidth}
      \begin{center}
        \includegraphics[width=\textwidth]{plot_arima_z_plot.png}
        \small 원시계열 + 조건부 저강수 월 표시
      \end{center}
  \end{columns}
\end{frame}

%===========================================================
\section{계절더미 + ARMA 모형}
%===========================================================



\begin{frame}{모형 기반 조건부 저강수 월 예시}
  \begin{columns}[T]
    \column{0.48\textwidth}
      \begin{center}
        \includegraphics[width=\textwidth]{plot_arma_z.png}
        \small $z_t$ 시계열과 $-k$ 기준선
      \end{center}
    \column{0.48\textwidth}
      \begin{center}
        \includegraphics[width=\textwidth]{plot_arma_z_plot.png}
        \small 원시계열 + 조건부 저강수 월 표시
      \end{center}
  \end{columns}
\end{frame}

%===========================================================
\section{논의 및 향후 과제}
%===========================================================

\begin{frame}{현재까지의 정리}
  \begin{itemize}
    \item 기본 모형: ARIMA(2,0,2)(0,1,1)[12]가 AIC/BIC와 잔차진단을 기준으로
          \alert{무난한 시계열 모형}으로 보임
    \item tsoutliers를 통해 \alert{5개의 시계열 이상치}를 자동 탐지
    \item 모형 기반으로 조건부 표준화 잔차 $z_t$를 정의하고
          \alert{조건부 저강수 월} 개념을 도입
  \end{itemize}
\end{frame}

\begin{frame}{해결되지 않은 문제와 향후 방향}
  \begin{itemize}
    \item \textbf{이상치 처리와 모형 재적합}
      \begin{itemize}
        \item tsoutliers로 탐지된 이상치를
              실제로 교정(replacement)한 후
              모형을 다시 적합하는지 여부는 아직 실험 단계
      \end{itemize}
    \item \textbf{분포 가정과 변환}
      \begin{itemize}
        \item 강수량은 비대칭/heavy-tail 분포를 가질 수 있음
      \end{itemize}
  \end{itemize}
\end{frame}

%===========================================================
\end{document}
%===========================================================